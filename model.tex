\section{Programmation sur GPU.}
%1
\begin{frame}
    \frametitle{Architecture matérielle.}
\begin{block}{Les multiprocesseurs de flux (SM).}
   \begin{itemize}
    \item<+-> Héritier du pipeline graphique, le multiprocesseur de flux ("Streaming multiprocessor", SM) 
    est une entité de traitement comportant un séquenceur, plusieurs unités de traitement numérique et une mémoire locale.
    \item<+-> Un processeur graphique (GPU) regroupe plusieurs multiprocesseurs. 
    \item<+-> Les multiprocesseurs exécutent des blocs de processus de façon \textbf{indépendante} et peuvent accéder à
    une mémoire partagée.
    \item<+-> Pour un développeur sur une architecture conventionnelle, un multiprocesseur s'apparente à un c{\oe}ur de calcul.
   \end{itemize} 
\end{block}
    

\end{frame}
%2
\begin{frame}
    \frametitle{Architecture matérielle.}
\begin{block}{Les multiprocesseurs de flux (SM).}
   \begin{itemize}
    \item<+->À l'intérieur d'un multiprocesseur, les processus s'exécutent de façon concurrente, mais peuvent communiquer
    via la mémoire locale ou être synchronisés.
    \item<+->Les processus sont regroupés par blocs, appelés "warps", qui se voient affecter le même séquenceur 
    d'instructions.
    \item<+->Le modèle associé est dit "SIMT" pour "Single Instruction Multiple Thread".
    \item<+->Les dernières générations de processeurs graphiques tendent à supprimer ces limitations.
   \end{itemize} 
\end{block}
\end{frame}
%3