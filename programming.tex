\section{Programmation}
\begin{frame}
    \frametitle{Compilation}
\begin{block}{Un dialecte de C/C++}
    \begin{itemize}
        \item<+-> Le GPU se programme en C/C++ avec des directives spécifiques reconnues par nvcc, le 
        compilateur de NVIDIA.
        \item<+-> Placées devant un nom de variable ou de fonction, elles permettent d'en spécifier l'emplacement.
        \item<+-> Le fichier devant être compilé avec nvcc doit avoir l'extension \texttt{.cu}.
        \item<+-> CMake reconnaît CUDA comme un langage à part entière, détecte l'environnement de développement NVIDIA
        et génère les projets en conséquence.
    \end{itemize}
\end{block}
\end{frame}
\begin{frame}
    \frametitle{Directives d'exécution}
\begin{block}{Avant la déclaration d'une fonction}
    \renewcommand{\arraystretch}{3}
    \vskip 20pt
    \begin{tabular}{|c|c|}
        \hline
        \rowcolor{lightgray} Directive & Effet \\ \hline
        \_\_global\_\_ & \begin{minipage}{0.8\textwidth}
            La fonction est un noyau. Elle est compilée pour le GPU, mais peut être appelée depuis le CPU. 
        \end{minipage} \\ \hline
        \_\_device\_\_ & \begin{minipage}{0.8\textwidth}
            La fonction est compilée pour le GPU et ne peut être appelée que depuis le GPU. 
        \end{minipage} \\ \hline
        \_\_host\_\_ & \begin{minipage}{0.8\textwidth}
            La fonction est compilée pour le CPU et ne peut être appelée que depuis le CPU (comportement par défaut). 
        \end{minipage} \\ \hline
    \end{tabular}
\end{block}

\end{frame}
\begin{frame}
    \frametitle{Directives de stockage}
\begin{block}{Avant la déclaration d'une variable}
    \renewcommand{\arraystretch}{2.5}
    \vskip 20pt
    \begin{tabular}{|c|c|}
        \hline
        \rowcolor{lightgray} Directive & Effet \\ \hline
        \_\_device\_\_ & \begin{minipage}{0.8\textwidth}
            La variable est stockée dans la mémoire globale. 
        \end{minipage} \\ \hline
        \_\_constant\_\_ & \begin{minipage}{0.8\textwidth}
            La variable est stockée dans la mémoire globale des constantes.
        \end{minipage} \\ \hline
        \_\_shared\_\_ & \begin{minipage}{0.8\textwidth}
            La variable est stockée dans la mémoire partagée d'un multiprocesseur.
        \end{minipage} \\ \hline
    \end{tabular}
\end{block}
\end{frame}

\begin{frame}
    \frametitle{Directives d'exécution}
\begin{block}{Lancement d'un noyau}
    \begin{itemize}
          \item<+-> Une fonction \texttt{fun} déclarée avec la directive \texttt{\_\_global\_\_} est éligible à une exécution parallèle 
  sur le GPU.
  \item<+-> La directive de lancement prend la forme suivante:
  \begin{verbatim}
    fun<<<grid_dim,bloc_dim,
  \end{verbatim}
    \end{itemize}

\end{block}

\end{frame}
